\documentclass[11pt]{article}
\usepackage{hyperref}
\urlstyle{same}
\usepackage{amsmath}
\usepackage{minted}
\usemintedstyle{tango}
\usepackage{inconsolata}
%\usepackage[T1]{fontenc}
\setcounter{secnumdepth}{-1}

\title{Programming language communities on GitHub}
\author{Corey Ford}

\begin{document}
\maketitle

\section{Introduction}

The website GitHub facilitates collaborative software development and hosts the
source code repositories, as well as other development activities, of numerous
open-source software projects, many of them widely-known.

To what extent does the diversity of programming language fragment the software
development community? We anticipate that the frequency of collaboration between
projects using the same programming language (or software framework, especially)
will result in communities identified by a programming language.

In this paper, we construct the one-mode network of repositories and examine its
community structure, comparing communities defined by programming languages to
those determined by common community-detection algorithms.

\section{Related Work}
Some recent analyses have examined the programming languages associated with
each user via their repositories to determine conditional probabilities
\cite{doll12} or correlations \cite{shah13} relating to use of language pairs.

User communities based on primary programming languages and ``follower''
relationships were explored in \cite{cuny10,weber12}.  A visualization of the
bipartite graph of organizations (groups of users) and languages
\cite{rodrigues12} won third place in the 2012 GitHub Data Challenge.

Methods for generating the the project-project and user-user networks are
described in \cite{thung2013} and used to find nodes with high PageRank values.
A weighted community detection algorithm for use in evaluating the GitHub
network is presented in \cite{marrama}.

\section{Methodology}
Data was obtained using SQL queries against the Google BigQuery web interface.
Further analysis was performed using Python 2.7 and version 0.6.5 of the igraph
library, and visualizations were created using Gephi 0.8.2 and sigma.js. All SQL
and Python code, along with data files, is available at
\url{https://github.com/coyotebush/github-network-analysis}.

\section{Data Collection}
We queried the GitHub Archive dataset\cite{githubarchive} to obtain data on
GitHub repositories that had received at least 1000 ``stars'' (a way for users
to bookmark interesting repositories) as of the end of 2012, including the URL,
primary programming language, and number of stars for each repository.

Using a second query, we computed a set of weighted edges between these
repositories. This weight is based on two types of events: \emph{pushes}, where
a user adds new changes to a repository, and \emph{pull-request merges}, where a
user without push access proposes changes that are then reviewed and added to
the repository by another user who does.\footnote{Accounting only for pushes gave a relatively disconnected graph,
presumably because, as examined in \cite{khadke}, most users have push access to only a few related
repositories.} We first computed contribution weights $C_{ui}$ for
user-repository pairs as the number of times the user has either pushed to the
repository or had a pull-request accepted to the repository during
2012.\footnote{This counts pull-request merges for both the user who submitted
the request and the user who pushed the merge; this seems acceptable, since it
accounts for the work of reviewing a pull-request and better identifies
repository maintainers.}

Weights between repositories were then computed by examining users who have
contributed to each of a pair of repositories, and defining the weight of the
edge between repositories $i$ and $j$ as the sum over all users who have
contributed to both of the geometric mean of the number of times they have
contributed to each. The geometric mean was chosen because it is high when the
counts of a user's contributions to both repositories are high, but low when
either is low (and elegantly zero when they have not contributed at all to
either), and the impact of increasing either is diminishing. That is, the the
weights are defined by $W_{ij} \equiv \sum_u \sqrt{C_{ui} \cdot C_{uj}}$ , where
$W$ gives the edge weight, $C$ is the contribution weight defined previously,
and $u$ ranges over all GitHub users \footnote{This is similar to the approach
of \cite{marrama}, which effectively used $\sqrt{min(C_{ui}, C_uj)}$;
\cite{opsahlproj,opsahl11} provide additional insight into projections of
such two-mode networks.}

The resulting graph contains 825 nodes and 3661 edges. However, for the
remainder of the analysis we consider only the giant component of this graph,
which contains 632 nodes and 3644 edges. This is justified by the lack of
meaningful structure in the remainder of the graph---the remaining connected
components comprise at most 3 nodes each---and reduces clutter both in the
visualization (which would have many nodes floating at the edge) and the
community detection (which would identify most of these nodes as their own
communities).

\begin{itemize}
    \item Visualize the network, with nodes colored by language
    \item Determine the modularity (and similar measures of community structure)
        of each of the language subnetworks.
    \item Apply common community-finding algorithms to the network and analyze
        the programming language composition of the resulting structure.
\end{itemize}

\section{Visualization}
A visualization of the network using the ForceAtlas 2 layout algorithm, with
nodes colored according to programming language and sized according to number of
stars, is shown in figure~\ref{fig:fullnetwork}. An interactive version is
available at \url{http://coyotebush.github.io/github-network-analysis/}.

\begin{figure}[htbp]
    \centering
    (Visualization goes here)
    \caption{GitHub repository network.}
    \label{fig:fullnetwork}
\end{figure}

Some language-based communities are already visually evident, including Ruby
(pink), JavaScript (pale blue), Python (yellow) and Objective-C (dark blue).

\section[Analysis]{Analysis\footnote{This section was created using Pweave:
\url{http://mpastell.com/pweave/}}}

\input{analysis.tex}

\section{Discussion}

\section{Conclusions}

\section{Acknowledgements}
This project is being developed both as the final project for a course in social
network analysis \cite{snacourse}, and as an entry in the GitHub Data Challenge
\cite{doll13}.

Thanks to Peter Faiman and Corey Farwell for help regarding SQL and
visualization, respectively.

\bibliographystyle{plain}
\bibliography{github}

\end{document}
% vim: ts=4 sts=4 sw=4 et tw=80
