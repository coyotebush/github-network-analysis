\documentclass[twocolumn]{article}
\usepackage{fullpage}
\usepackage{hyperref}

\title{Programming language communities on GitHub}
\author{Corey Ford}

\begin{document}
\maketitle

\section*{Introduction}

GitHub \cite{github} facilitates collaborative software development and hosts
the source code repositories, as well as other development activities, of many
open-source software projects small and large. Data on the repositories hosted
at GitHub, the users of the service, and the changes (or \textit{commits}) users
make to repositories is available through an HTTP API as well as through Google
BigQuery.

We examine the relationship between the community structure of the GitHub
collaboration network and the programming languages used for projects. The
question might be stated as: to what extent does the diversity of programming
language fragment the software development community? We anticipate that
collaborations between projects using the same programming language (or
especially the same software framework) will be common, resulting in communities
identified by a programming language.

Some recent work has examined the programming languages associated with each
user via their repositories to determine conditional probabilities \cite{doll12}
or correlations \cite{shah13} relating to the use of language pairs. One winner
of the 2012 GitHub Data Challenge created a visualization of the bipartite graph
of organizations (groups of users) and languages \cite{rodrigues12}. We approach
the problem somewhat differently by constructing a graph of relationships
between repositories and then examining its community structure.

Methods for generating the the project-project and user-user networks are
described in \cite{thung2013} and used to find nodes with high PageRank values.
A weighted community detection algorithm for use in evaluating the GitHub
network is described in \cite{marrama}. Other analysis of the GitHub network
includes examination of different types of collaboration \cite{khadke} and
ranking and recommendation of developers \cite{sarma}.

\section*{Methodology}
\begin{itemize}
    \item Use Google BigQuery to obtain data on the top $n$ GitHub repositories
        and all commits made to them, including data on the primary programming
        language used in each repository and the user who made each commit
        (where $n$ is small enough to make computation and visualization
        feasible).
    \item Transform this data into a weighted network of repositories
    \item Visualize the network, with nodes colored by language
    \item Determine the modularity (and similar measures of community structure)
        of each of the language subnetworks.
    \item Apply common community-finding algorithms to the network and analyze
        the programming language composition of the resulting structure.
\end{itemize}

\section*{Acknowledgements}
This project is being developed both as the final project for a course in social
network analysis \cite{snacourse}, and as an entry in the GitHub Data Challenge
\cite{doll13}.

%\begin{thebibliography}{9}
%\bibitem{snacourse}
    %Adamic, Lada. ``Social Network Analysis''.
    %\url{https://www.coursera.org/course/sna}
%\bibitem{doll12}
    %Doll, Brian. ``Programming Language Correlation Dataset''.
    %\url{https://gist.github.com/briandoll/e0637fff9c8eec988528}
%\bibitem{doll13}
    %Doll, Brian. ``The GitHub Data Challenge II''.
    %\url{https://github.com/blog/1450}
%\bibitem{github}
    %GitHub. \url{https://github.com/}
%\bibitem{rodrigues12}
    %Rodrigues, Eduarda Mendes. ``GitHub Data''.
    %\url{http://labs.sapo.pt/networds/2012/05/09/github-data/}
%\bibitem{shah13}
    %Shah, Akshay. ``Language Use on GitHub''.
    %\url{http://datahackermd.com/2013/language-use-on-github/}
%\end{thebibliography}

\bibliographystyle{plain}
\bibliography{github}
\end{document}
% vim: ts=4 sts=4 sw=4 et tw=80
