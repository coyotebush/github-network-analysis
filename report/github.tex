\documentclass[twocolumn]{article}
\usepackage{fullpage}
\usepackage{hyperref}
\usepackage{amsmath}
\usepackage{courier}
%\usepackage[T1]{fontenc}
\usepackage{listings}
\lstset{
    basicstyle=\ttfamily\small,
    keywordstyle=\bfseries,
    breaklines=true
}

\title{Programming language communities on GitHub}
\author{Corey Ford}

\begin{document}
\maketitle

\section{Introduction}

GitHub \cite{github} facilitates collaborative software development and hosts
the source code repositories, as well as other development activities, of many
open-source software projects small and large. Data on the repositories hosted
at GitHub, the users of the service, and the changes (or \textit{commits}) users
make to repositories is available through an HTTP API as well as through Google
BigQuery.

We examine the relationship between the community structure of the GitHub
collaboration network and the programming languages used for projects. The
question might be stated as: to what extent does the diversity of programming
language fragment the software development community? We anticipate that
collaborations between projects using the same programming language (or
especially the same software framework) will be common, resulting in communities
identified by a programming language.

Some recent analyses have examined the programming languages associated with each
user via their repositories to determine conditional probabilities \cite{doll12}
or correlations \cite{shah13} relating to use of language pairs. A visualization
of the bipartite graph of organizations (groups of users) and languages
\cite{rodrigues12} won third place in the 2012 GitHub Data Challenge . We
approach the problem somewhat differently by constructing the one-mode network
of repositories and examining its community structure.

Methods for generating the the project-project and user-user networks are
described in \cite{thung2013} and used to find nodes with high PageRank values.
A weighted community detection algorithm for use in evaluating the GitHub
network is described in \cite{marrama}.

\section{Methodology}
We used Google BigQuery on the GitHub archive to obtain data on GitHub
repositories that had received at least 1000 ``stars'' as of the end of 2012,
including the URL, primary programming language, and number of stars for each
repository.

Using a second query, we computed a set of weighted edges between these
repositories. This weight is based on two types of events: \emph{pushes}, where
a user adds new changes to a repository, and \emph{pull-request merges}, where a
user without push access proposes changes that are then reviewed and added to
the repository by another user who does.\footnote{In preliminary analysis,
accounting only for pushes resulted in a relatively disconnected graph,
presumably because, as examined in \cite{khadke}, most users have push access to only a few related
repositories.} We first computed contribution weights $C(u, r)$ for
user-repository pairs as the number of times the user has either pushed to the
repository or had a pull-request accepted to the repository during
2012.\footnote{This counts pull-request merges for both the user who submitted
the request and the user who pushed the merge; this seems acceptable, since it
accounts for the work of reviewing a pull-request and better identifies
repository maintainers.}

Weights between repositories were then computed by examining users who have
contributed to each of a pair of repositories, and defining the weight between
repositories as the sum of all these user-repository weights. That is, the
resulting 100+-line SQL query could be described as

\begin{equation}
    W(r1, r2) \equiv \sum_{u \in U} C(u, r1) ??? C(u, r2)
\end{equation}

where $W$ gives the weight of the edge between two repositories, $C$ is the
contribution weight defined previously, and $U$ is the set of all GitHub users
who have contributed to any of these top repositories.

Code for these queries is included in appendix~\ref{app:sql}.

\begin{itemize}
    \item Visualize the network, with nodes colored by language
    \item Determine the modularity (and similar measures of community structure)
        of each of the language subnetworks.
    \item Apply common community-finding algorithms to the network and analyze
        the programming language composition of the resulting structure.
\end{itemize}

\section{Acknowledgements}
This project is being developed both as the final project for a course in social
network analysis \cite{snacourse}, and as an entry in the GitHub Data Challenge
\cite{doll13}.

\bibliographystyle{plain}
\bibliography{github}

\onecolumn
\appendix
\section{Source code}
\subsection{SQL queries}
\label{app:sql}

\subsubsection{\texttt{github-repo-attributes.sql}: Fetch data on top
repositories}
\lstinputlisting[language=SQL]{../github-repo-attributes.sql}

\subsubsection{\texttt{github-repo-weights.sql}: Determine weights between top
repositories}
\lstinputlisting[language=SQL]{../github-repo-weights.sql}

\end{document}
% vim: ts=4 sts=4 sw=4 et tw=80
